\documentclass[a4paper,12pt]{article} % тип документа

% Поля страниц
\usepackage[left=2.5cm,right=2.5cm, top=2cm,bottom=2cm,bindingoffset=0cm]{geometry}
    
%Пакет дял таблиц   
\usepackage{multirow} 

    
%Отступ после заголовка    
\usepackage{indentfirst}


% Рисунки
\usepackage{subcaption,floatrow,graphicx,calc}
\usepackage{wrapfig}

% Создаёем новый разделитель
\DeclareFloatSeparators{mysep}{\hspace{1cm}}

% Ссылки?
\usepackage{hyperref}
\usepackage[rgb]{xcolor}
\hypersetup{				% Гиперссылки
    colorlinks=true,       	% false: ссылки в рамках
	urlcolor=blue          % на URL
}


%  Русский язык
\usepackage[T2A]{fontenc}			% кодировка
\usepackage[utf8]{inputenc}			% кодировка исходного текста
\usepackage[english,russian]{babel}	% локализация и переносы


% Математика
\usepackage{amsmath,amsfonts,amssymb,amsthm,mathtools, mathrsfs, wasysym}

\newcommand{\const}{\mathrm{const}}
\newcommand{\rref}[1]{(\ref{#1})}
\newenvironment{comment}{}{}
\newcommand{\picref}[1]{рис. \ref{#1}}
\newcommand{\mbf}{\mathbf}
\newcommand{\Equip}[3]{
	
	\item{\bf #1:} $\Delta = \pm #2\; #3$}
\newcommand{\equip}[1]{
\item{\bf #1}}



\begin{document}

\begin{titlepage}
	\centering
	\vspace{5cm}
	{\scshape\LARGE Московский физико-технический институт \par}
	\vspace{4cm}
	{\scshape\Large Лабораторная работа № 11а \par}
        {\scshape\Large по курсу \par}
        {\scshape\Large Твердотельная электроника \par}
	\vspace{1cm}
	{\huge\bfseries  Фотоэлектрический способ преобразования энергии солнечного излучения \par}
	\vspace{1cm}
	\vfill
\begin{flushright}
	{\large выполнили студенты Б04-004 группы}\par
	\vspace{0.3cm}
	{\LARGE Семёнова Наталия}\par
        \vspace{0.3cm}
	{\LARGE Атласов Владислав}\par
        \vspace{0.3cm}
	{\LARGE Плюскова Наталия}\par
        \vspace{0.3cm}
	{\LARGE Богатова Екатерина}
\end{flushright}

	\vfill
	
% Bottom of the page
	Долгопрудный, 2023 г.
\end{titlepage}

\textbf{В работе:} Исследование темновой и световой вольтамперной характеристики фотоэлемента. Изучение влияния мощности падающего излучения на характеристики образца с помощью фильтров.

\section{Теоретическое введение}
Электронно-дырочный переход\\
Прямое преобразование лучистой энергии Солнца в электрическую осуществляется с помощью фотоэффекта на потенциальном барьере или так называемого вентильного фотоэффекта, суть которого – возникновение фото-ЭДС при освещении контактов металл-полупроводник и p-n переходов. Однако, вследствие сложной микроструктуры контактов полупроводника с металлом, мы ограничимся в дальнейшем наиболее ясным случаем p-n переходов.\\
Рассмотрим более подробно, что представляет собой p-n переход. Пусть два полупроводника, один из которых имеет проводимость p-типа, а другой n-типа приводятся в хороший контакт по плоскости $aa$', как показано на изображении а (Рис. 1). Тогда под действием градиента концентрации дырки из приконтактного слоя p-области будут диффундировать в n-область, а электроны из приконтактного слоя n-области в p-область.

\begin{figure}[h]
		\centering	
		\includegraphics[width=0.45\textwidth]{pnper.png}
		\caption{p-n переход}
		\label{pic:scheme}
	\end{figure}

В результате такой диффузии в приконтактном слое p-области создается отрицательный объемный заряд нескомпенсированных ионов акцепторной примеси, а в приконтактном слое n-области – положительный объемный заряд нескомпенсированных ионов донорной примеси. Порожденное объемными зарядами электрическое поле (направление которого показано на изображении а (Рис. 1)), будет препятствовать дальнейшей диффузии основных носителей зарядов (основными называются носители, знак которых соотвествует типу проводимости полупроводника). При этом напряженность электрического поля $\epsilon$ и толщины слоев объемных зарядов в n и p-областях будут возрастать до тех пор, пока не достигнут своих равновесных значений $\epsilon_0$, $d_p$ и $d_n$, при которых диффузионные потоки основных носителей зарядов полностью скомпенсированы дрейфовыми потоками, вызванными электрическим полем объемных зарядов. Переходная область толщины $d=d_n+d_p$, объединённая свободными носителями зарядов, и в которой локализовано электрическое поле с напряженностью $\epsilon _0 $ получила название электронно-дырочного или p-n− перехода. Толщина p-n-перехода d для различных полупроводниковых систем может изменяться от единицы и до сотых долей микрометров, а величина $\epsilon_0$ достигать значений $\sim 10^7 B*cm^{-1}$.\\

Состояние p-n-перехода в термодинамическом равновесии легко понять, обращаясь к его энергетической диаграмме, приведенной на cхеме б (Рис. 1). Здесь $E_c$ – дно зоны проводимости, $E_\nu$ – потолок валентной зоны, F – уровень Ферми. В самом деле, электроны из n-области не могут проникнуть в p-область, так как для этого им необходимо преодолеть потенциальный барьер, высота $u_k$ которого равна контактной разности потенциалов, а энергия электронов меньше высоты этого барьера. По аналогичной причине дырки из p-области не могут попасть в n-область.\\

На практике p-n-переходы реализуются не механическим соединением двух полупроводников, а внутри единого кристалла, в котором создают подходящее распределение донорной $N_d$ и акцепторной $N_a$ примесей, например, показанной на схеме в (Рисунок 1).\\
\section{Экспериментальная установка}
Вольтамперная характеристики фотопреобразователя могут быть измерены с помощью схемы, представленной на схеме (Рис. 2). Когда преобразователь работает как генератор электроэнергии, то в качестве источника излучения используется лампа марки 3H7 или 3Н8 с встроенным зеркальным отражателем и мощностью 500 Вт. Спектр ее излучения с помощью водяного фильтра приближен к спектру солнечного излучения и к спектральной чувствительности кремниевого преобразователя.\\


\begin{figure}[!h]
		\centering	
		\includegraphics[width=0.6\textwidth]{ust.jpg}
		\caption{Схема установки для экспериментальных исследований световой и темновой в/а характеристик фотоэлементов}
		\label{pic:scheme}
	\end{figure}

Перед началом и после измерений тумблеры на пульте управления и переключатели других приборов должны быть установлены в следующих положениях:\\
- Тумблер 1: подачи напряжения от батареи аккумуляторов – в положении «выкл».\\
- Тумблер 2 – переключение полярности микроамперметра М 95 – в положении «обр».\\
- Тумблер 3 – переключение с микроамперметра на миллиамперметр – в положении «мА».\\
- Тумблер 4 – изменение полярности напряжения подаваемого на фотопреобразователь – в положении «обр.».\\
Потенциометры «грубо» и «точно» должны быть выведены против часовой стрелки до упора.\\
Переключатель шкалы микроамперметра М 95 должен стоять в положении «арретир», а наружный шунт к М 95 в положении «$\infty$».\\

\section{Ход работы}

Проведём измерение темновых вольтамперных харакстерстик исследуемого кремниевого фотоэлемента.

Начнём с прямой ветви темновой в/а характеристики. Будем снимать точки до 300 мВ с шагом в 20 мВ, а затем до 700 мВ с шагом 50 мВ. Предварительно не забудем учесть, что ноль напряжений смещён на 0.9 делений (на 9 мВ) вправо.

\begin{table}[H]
\centering
\caption{Зависимость значений прямого темнового тока от напряжения на фотоэлементе}
\begin{tabular}{|l|l|l|l|l|l|l|l|l|l|l|l|}
\hline
U, мВ  & 21   & 41  & 61  & 81   & 101  & 121  & 141  & 161   & 181   & 201 & 221   \\ \hline
I, мкА & 4,9  & 9   & 13  & 17   & 21   & 24   & 28   & 32,5  & 36    & 40  & 44    \\ \hline
U, мВ  & 241  & 261 & 281 & 331  & 381  & 431  & 481  & 531   & 581   & 631 & 681   \\ \hline
I, мкА & 48,3 & 50  & 54  & 62,5 & 72,5 & 82,5 & 92,5 & 102,5 & 111,5 & 121 & 130,5 \\ \hline
\end{tabular}
\label{tab:direct}
\end{table}

Основываясь на данных из таблицы 
\ref{tab:direct} построим график зависимости тока от напряжения (вольт-амперную характеристику) -- рис \ref{fig:direct} -- и фиттируем её линейной зависимость, проходящей через ноль.

\begin{figure}[H]
    \centering
    \includegraphics[scale = 0.7]{Direct.png}
    \caption{Вольт-амперная характеристика темнового тока (прямая ветвь)}
    \label{fig:direct}
\end{figure}

Здесь в качестве погрешностей использовали цену деления прибора: 10 мВ для напряжения и 5 мкА для тока.

Теперь аналогично снимем обратную ветвь темновой в/а характеристики. Будем снимать точки с шагом в 10 мВ до значения напряжения в 100 мВ. Также учтём, что напряжения сдвинуто вправо на 9 мВ.

\begin{table}[H]
\centering
\caption{Зависимость значений обратного темнового тока от напряжения на фотоэлементе}
\begin{tabular}{|l|l|l|l|l|l|l|l|l|l|l|l|}
\hline
U, мВ  & 1 & 11 & 21 & 31  & 41  & 51   & 61   & 71   & 81 & 91 & 101  \\ \hline
I, мкА & 0 & 2  & 4  & 6,3 & 8,2 & 10,1 & 12,5 & 14,7 & 17 & 19 & 20,5 \\ \hline
\end{tabular}
\label{tab:reverse}
\end{table}

Из таблицы \ref{tab:reverse} строим ВАХ уже для обратной ветви темнового тока -- рис. \ref{fig:reverse}

\begin{figure}[H]
    \centering
    \includegraphics[scale = 0.7]{Reverse.png}
    \caption{Вольт-амперная характеристика темнового тока (обратная ветвь)}
    \label{fig:reverse}
\end{figure}

Здесь погрешность для напряжения -- 5 мВ, для тока -- 1 мкА (цена деления для тока изменилась в случае обратной ветви).

Теперь оценим $R_\text{ш}$ и $R_\text{п}$ как обратные угловые коэффициенты на рисунках \ref{fig:reverse} и \ref{fig:direct} соответсвенно.

\[R_\text{ш} = \left( \dfrac{dI}{dU} \right)^{-1}_\text{обр}; \quad R_\text{п} = \left( \dfrac{dI}{dU} \right)^{-1}_\text{прям}. \]

Погрешности оценим по следующей формуле (как погрешность косвенной величины $\sigma_y = \left| \dfrac{dy}{dx} \right| \sigma_x$):
\[ \sigma_R = \dfrac{\sigma_{(dI/dU)}}{\dfrac{dI}{dU}}.\]

Таким образом, получаем следующий результат:

\[R_\text{ш} = (4.75 \pm 0.01) \cdot 10^3 \, \text{Ом}; \quad  R_\text{п} = (5.25 \pm 0.02) \cdot 10^3 \, \text{Ом}.\]

Теперь построим график зависимости $ln(I) = f \left( U \right)$ для прямой ветви тока, чтобы найти коэффициенты $A$ и $I_s$.

\begin{figure}[H]
    \centering
    \includegraphics[scale=0.7]{Ln_direct.png}
    \caption{График зависимости $ln(I) = f \left( U \right)$ (для прямой ветви темнового тока)}
    \label{fig:ln_direct}
\end{figure}

Погрешность для $ln(I)$ оценили по формуле для погрешности косвенного измерения:
\[ \sigma_{ln(I)} = \dfrac{\sigma_I}{I}.\]

Коэффициенты $A$ и $I_s$ Найдём с помощью следующей формулы:
\begin{equation}\label{eq:A_and_Is}
    ln(I) = ln(I_s) + U \dfrac{e}{A \cdot kT}.
\end{equation}

Таким образом, отсюда и рис. \ref{fig:ln_direct} находим коэффициенты $A$ и $I_s$:

\[ I_s = e^b = (17 \pm 5) \, \text{мкА} \]
\[ A = \dfrac{1}{0.025 \cdot k} = (12 \pm 4)\]

\section{Вывод}

В ходе данной лабораторной работы мы сняли вольт-амперные характеристики темнового тока для прямой и обратных ветвей для исследуемого кремниевого фотоэлемента (рис. \ref{fig:direct}, \ref{fig:reverse}). С помощью этих вольт-амперных характеристик рассчитали значения для $R_\text{ш} = (4.75 \pm 0.01) \cdot 10^3$ Ом и $R_\text{п} = (5.25 \pm 0.02) \cdot 10^3$ Ом. Также, с помощью графика зависимости $ln(I) = f \left( U \right)$ (рис. \ref{fig:ln_direct}) для прямой ветви темнового тока рассчитали коэффициенты $A$ и $I_s$, фигурирующие в формуле \ref{eq:A_and_Is}: $I_s = (17 \pm 5) \, \text{мкА}$, $A = (12 \pm 4)$. 

График зависимости $ln(I) = f \left( U \right)$ (рис. \ref{fig:ln_direct}) получился очень не точным (плохо фиттируется линейной зависимостью). Это может быть связано с проблемами, возникшими в ходе выполнения работы. Например, у нас мог сгореть фотоэлемент, из-за чего значения, полученные нами, оказались неточными.

\end{document}
