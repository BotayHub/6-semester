\documentclass[a4paper,12pt]{report}
\usepackage[T2A]{fontenc}
\usepackage[utf8]{inputenc}
\usepackage[english,russian]{babel}
\usepackage{graphicx}
\usepackage{wrapfig}
\usepackage{mathtext} 				% русские буквы в фомулах
\usepackage{amsmath,amsfonts,amssymb,amsthm,mathtools} % AMS
\usepackage{icomma} % "Умная" запятая: $0,2$ --- число, $0, 2$ --- перечисление
\usepackage{capt-of}
\usepackage{appendix}
\usepackage{multirow}
\usepackage{hyperref}
\usepackage{floatrow}
\usepackage[left=2cm,right=2cm,
    top=2cm,bottom=2cm,bindingoffset=0cm]{geometry}
\usepackage{multicol} % Несколько колонок
\usepackage{gensymb}
\title{Отчёт по лабораторной работе №10

ТЕМА}
\author{Плюскова Н., \\
Богатова Е., \\
Атласов В., \\
Соколов А.}
\date{\today}

\begin{document}

\maketitle


\section*{1. Теоретические данные}

В работе изучается влияние сильного электрического поля на электропроводность полупроводников, а также эффекты, связанные с возникновением отрицательной дифференциальной проводимости при разогреве электронов.

Эффект разогрева электронов заключается в том, что при воздействии на свободные электроны внешних сил, например электрического поля, средняя энергия электронного газа может существенно превысить своё равновесное значение, в то время как энергия решётки будет оставаться почти без изменения. Это явление характерно для сравнительно слабо легированных полупроводников ($n \approx 10^{17}$ см$^{-3}$), в этом случае теплоёмкость решётки намного превышает теплоёмкость электронного газа, и её можно рассматривать как термостат с температурой, не зависящей от электрического поля.

При наличии достаточно частых межэлектронных столкновений функция распределения близка к распределению Максвелла-Больцмана с температурой $T_{e}$. Изменение вида функции распределения в сильном электрическом поле приводит к тому, что подвижность электроном становится функцией электрического поля. В частности, можно показать, что подвижность является степенной функцией поля: в случае рассеивания на акустических фононах $\mu \sim E^{-1/2}$, в случае рассеивания на оптических фононах возможна зависимость $\mu \sim E^{-1}$.

Полупроводниковые кристаллы обладают анизотропией, в пределах зоны Бриллюэна есть несколько минимумов зоны проводимости, которые характеризуются различной эффективной массой. В некоторых полупроводниках группы $A^{3}B^{5}$, в частности GaAs, более высоко расположенные минимумы характеризуются большей эффективной массой, чем низко расположенные. Поэтому когда электроны, разогретые полем, становятся способными перейти из нижней долины в верхнюю, они «тяжелеют», подвижность их уменьшается, это приводит к уменьшению проводимости в сильном поле. С увеличением приложенного
поля вероятность перехода электронов из нижней долины в верхнюю увеличивается настолько, что, начиная с некоторого порогового значения поля дифференциальная проводимость становится отрицательной.

Следствием отрицательной проводимости является неустойчивость тока в полупроводнике. На участке отрицательной дифференциальной проводимости любая случайная флуктуация поля объёмного заряда в кристалле имеет тенденцию к нарастанию, возникают участки сильного поля в кристалле, называемые доменами. Домены могут быть статическими (при их возникновении ток в образце приходит к насыщению или даже уменьшается при увеличении напряжения) и динамические (при их возникновении начинается генерация периодических колебаний тока в образце).


\section*{2. Экспериментальная установка}
\begin{figure}[H]
    \centering
    \includegraphics[width=0.4\linewidth]{ustanovka_10.jpg}
    \caption{Схема экспериментальной установки} \label{ustanovka}
\end{figure}

\begin{figure}[H]
    \centering
    \begin{minipage}[H]{0.49\linewidth}
        \includegraphics[width = 1\linewidth]{Exp_setup_1.jpg}
        \label{fig:exp_set_1}
    \end{minipage}
    \begin{minipage}[H]{0.49\linewidth}
        \includegraphics[width = 1\linewidth]{Exp_setup_2.jpg} 
        \label{fig:exp_set_2}
    \end{minipage}
    \caption{Фотографии экспериментальной установки}
\end{figure}

\section*{3. Результаты эксперимента и обработка данных}
Собрав и настроив установку, получим зависимость координаты пучности от ее порядкового номера:

\begin{table}[H]
\begin{tabular}{|l|l|l|l|l|l|l|l|l|}
\hline
Номер пучности      & 1    & 2     & 3     & 4     & 5     & 6     & 7     & 8     \\ \hline
Значения координаты, мм & 9,25 & 13,85 & 18,10 & 22,05 & 26,45 & 30,25 & 34,75 & 39,10 \\ \hline
\end{tabular}
\caption{Координаты пучности и их порядковые номера}
\label{tab_1}
\end{table}

По данным таблицы \ref{tab_1} построим соответствующий график:

\begin{figure}[H]
    \centering
    \includegraphics[width = 0.8\linewidth]{Graph_1.png}
    \caption{Зависимость координаты пучности от её порядкового номера}
    \label{fig:graph_1}
\end{figure}

Из графика \ref{fig:graph_1} получим половину длины волны как коэффициент наклона и частоту генерации соответственно:

\[ \dfrac{\lambda}{2} = 4.22 \pm 0.05 \text{ мм} \Rightarrow  \lambda = 8.4 \pm 0.1 \text{ мм};\]
\[ \nu = \dfrac{c}{\lambda} \approx 35.5 \pm 0.4 \text{ ГГц} \]

Погрешность для $\dfrac{\lambda}{2}$ берём равной цене деления штангенциркуля, так как погрешность, связанная с фитом по МНК, получается много меньше.

Погрешность для $\lambda$ и $\nu$ рассчитываем как погрешность косвенного измерения. Для $\lambda$ она получается просто в два раза больше погрешности для $\dfrac{\lambda}{2}$, а для $\nu$: $\sigma_{\nu} = \dfrac{c}{\lambda^2} \sigma_{\lambda}$.

\\

Исследуем ВАХ диода Ганна и построим соответствующий график:

\begin{table}[H]
\begin{tabular}{|l|l|l|l|l|l|l|l|l|l|l|l}
\hline
U, мВ & 1778,9 & 1772,3 & 1761   & 1757,5 & 1748,3 & 1720,1 & 1661,4 & 1639   & 1592,5 & 1536,5 & \multicolumn{1}{l|}{1484}   \\ \hline
I, мА & 964,9  & 978,8  & 980,7  & 983,8  & 982,7  & 987,6  & 1002   & 1003,9 & 1008,9 & 1014,7 & \multicolumn{1}{l|}{1019,9} \\ \hline
logU  & 3,25   & 3,249  & 3,246  & 3,245  & 3,243  & 3,236  & 3,22   & 3,215  & 3,202  & 3,187  & \multicolumn{1}{l|}{3,171}  \\ \hline
logI  & 2,984  & 2,991  & 2,992  & 2,993  & 2,992  & 2,995  & 3,001  & 3,002  & 3,004  & 3,006  & \multicolumn{1}{l|}{3,009}  \\ \hline
U, мВ & 1388   & 1272,2 & 1193,5 & 1055,5 & 963    & 849    & 725    & 642,4  & 519    & 420    &                             \\ \cline{1-11}
I, мА & 1027,5 & 1034,8 & 1033,8 & 1008,7 & 944,6  & 865    & 779,8  & 705    & 584    & 283,1  &                             \\ \cline{1-11}
logU  & 3,142  & 3,105  & 3,077  & 3,023  & 2,984  & 2,929  & 2,86   & 2,808  & 2,715  & 2,623  &                             \\ \cline{1-11}
logI  & 3,012  & 3,015  & 3,014  & 3,004  & 2,975  & 2,937  & 2,892  & 2,848  & 2,766  & 2,452  &                             \\ \cline{1-11}
\end{tabular}
\caption{Данные напряжения и силы тока на диоде}
\label{tab_2}
\end{table}

\begin{figure}[H]
    \centering
    \includegraphics[width = 0.8\linewidth]{Graph_2.png}
    \caption{ВАХ диода Ганна}
    \label{fig:my_label}
\end{figure}

\section*{4. Вывод}
В работе были изучены влияние сильного электрического поля на электропроводность полупроводников, а также эффекты, связанные с возникновением отрицательной дифференциальной проводимости при разогреве электронов.
В ходе эксперимента была найдена частота генерации ($\nu = 35.5 \pm 0.2 \text{ ГГц}$) и построена ВАХ диода Ганна. Качественный вид ВАХ диода Ганна согласуется с теорией.

\end{document}
