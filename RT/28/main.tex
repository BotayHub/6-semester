\documentclass[a4paper,12pt]{report}
\usepackage[T2A]{fontenc}
\usepackage[utf8]{inputenc}
\usepackage[english,russian]{babel}
\usepackage{graphicx}
\usepackage{wrapfig}
\usepackage{mathtext} 				% русские буквы в фомулах
\usepackage{amsmath,amsfonts,amssymb,amsthm,mathtools} % AMS
\usepackage{icomma} % "Умная" запятая: $0,2$ --- число, $0, 2$ --- перечисление
\usepackage{capt-of}
\usepackage{appendix}
\usepackage{multirow}
\usepackage{hyperref}
\usepackage{floatrow}
\usepackage[left=2cm,right=2cm,
    top=2cm,bottom=2cm,bindingoffset=0cm]{geometry}
\usepackage{multicol} % Несколько колонок
\usepackage{gensymb}
\title{Отчёт по лабораторной работе №28

Усилитель на биполярных транзисторах}
\author{Плюскова Н.А. Б04-004 }
\date{\today}

\begin{document}

\maketitle

\section*{1. Результаты эксперимента}
\subsection*{1.1 Нестабилизированный усилитель}

Соберем схему со следующими параметрами:
\begin{itemize}
    \item $R_{\text{К}}$ = 2 кОм
    \item $C_{\text{Б}}$ = 0,33-0,39 мкФ
\end{itemize}

\begin{enumerate}
\item В схеме, указанной на рисунке к п.1 методички (см. стр. 26), найдем $h_{21\text{э}}$. При этом для корректности следующих измерений подберем $R_{\text{Б}}$ так, чтобы $U_{\text{КЭ}}\approx$ 5 В:
    \begin{itemize}
        \item $R_{\text{Б}}$ = 521 кОм
        \item $h_{21\text{э}}$ = 70
    \end{itemize}

   \item Для нестабилизированного усилителя с $R_{\text{И}} \approx R_{\text{К}}$ найдем $U_{\text{вых.макс}}, K_{u}, K_{e}, R_{\text{вх}}, f_{\text{н}}, f_{\text{в}}$:

    При $R_{\text{Б}}$ = 521 кОм:
    \begin{itemize}
        \item $U_{\text{вых.макс}}$ = 3,89 В
        \item $K_{u}$ = 129
        \item $K_{e}$ = 71
        \item $R_{\text{вх}}$ = 90 кОм
        \item $f_{\text{н}}$ = 280 Гц
        \item $f_{\text{в}}$ = 32 кГц
    \end{itemize}

    При $R_{\text{Б}}$ = 1001 кОм:
    \begin{itemize}
        \item $U_{\text{вых.макс}}$ = 1,85 В
        \item $K_{u}$ = 84
        \item $K_{e}$ = 52
        \item $R_{\text{вх}}$ = 166 кОм
        \item $f_{\text{н}}$ = 140 Гц
        \item $f_{\text{в}}$ = 65 кГц
    \end{itemize}


    \item Определим $U_{\text{вых.макс}}, K_{u}, K_{e}, R_{\text{вх}}, f_{\text{н}}, f_{\text{в}}$ нестабилизированного усилителя с внешней нагрузкой $R_{\text{Н}} \approx R_{\text{К}}$ при $R_{\text{Б}}$ = 521 кОм и $C_{\text{р}}$ = 220 мкФ:
    \begin{itemize}
        \item $U_{\text{вых.макс}}$ = 1,65 В
        \item $K_{u}$ = 79
        \item $K_{e}$ = 40
        \item $R_{\text{вх}}$ = 234 кОм
        \item $f_{\text{н}}$ = 253 Гц
        \item $f_{\text{в}}$ = 350 кГц
    \end{itemize}
    
\end{enumerate}

\subsection*{1.2 Стабилизированный усилитель}
\begin{enumerate}
    \item Подберем $R_{1}$ так, чтобы $I_{\text{К}}$ = 1,19 мА:
    \begin{itemize}
        \item $R_{\text{э}}$ = 0,4 кОм
        \item $R_{2}$ = 2 кОм
        \item $U_{\text{Б}}$ = 2,19 В
        \item $U_{\text{Э}}$ = 1,5 В
        \item $U_{\text{К}}$ = 5,2 В
    \end{itemize}
    \item Для стабилизированного усилителя с $C_{\text{Э}}$ = 220 мкФ измерить $U_{\text{вых.макс}}, K_{u}, K_{e}, R_{\text{вх}}, f_{\text{н}}, f_{\text{в}}$ при $R_{\text{И}} \approx R_{\text{К}}$:
    \begin{itemize}
        \item $U_{\text{вых.макс}}$ = 2,32 В
        \item $K_{u}$ = 121
        \item $K_{e}$ = 47
        \item $R_{\text{вх}}$ = 1,3 кОм
        \item $f_{\text{н}}$ = 430 Гц
        \item $f_{\text{в}}$ = 270 кГц
    \end{itemize}

    
\end{enumerate}

\end{document}
