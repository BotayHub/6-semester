\documentclass[15pt,a5paper,reqno]{article}
\usepackage{hyperref}
\usepackage[warn]{mathtext}
\usepackage[utf8]{inputenc}
\usepackage[T2A]{fontenc}
\usepackage[russian]{babel}
\usepackage{amssymb, amsmath, multicol}
\usepackage{graphicx}
\usepackage[shortcuts,cyremdash]{extdash}
\usepackage{wrapfig}
\usepackage{floatflt}
\usepackage{lipsum}
\usepackage{verbatim}
\usepackage{concmath}
\usepackage{euler}
\usepackage{xcolor}
\usepackage{etoolbox}
\usepackage{fancyhdr}
\usepackage{subfiles}
\usepackage{enumitem}
\usepackage{amsthm}
\usepackage{indentfirst}
\usepackage{import}

\DeclareMathOperator{\sign}{sign}

\RequirePackage[ left     = 1.5cm,
  right    = 1.5cm,
  top      = 2.0cm,
  bottom   = 1.25cm,
  includefoot,
  footskip = 1.25cm ]{geometry}
\setlength    {\parskip}        { .5em plus .15em minus .08em }
%\setlength    {\parindent}      { .0em }
\renewcommand {\baselinestretch}{ 1.07 }

\fancyhf{}

\renewcommand{\footrulewidth}{ .0em }
\fancyfoot[C]{\texttt{\textemdash~\thepage~\textemdash}}
\fancyhead[R]{\hfilДержавин, Хайдари, Шурыгин}

\makeatletter
\patchcmd\l@section{%
  \nobreak\hfil\nobreak
}{%
  \nobreak
  \leaders\hbox{%
    $\m@th \mkern \@dotsep mu\hbox{.}\mkern \@dotsep mu$%
  }%
  \hfill
  \nobreak
}{}{\errmessage{\noexpand\l@section could not be patched}}
\makeatother
\parindent = 1cm % отступ при красной строке⏎
\pagestyle{fancy}    
\renewcommand\qedsymbol{$\blacksquare$}

\newcommand{\when}[2]{
  \left. #1 \right|_{#2} \hspace
}
\renewcommand{\kappa}{\varkappa}
\RequirePackage{caption2}
\renewcommand\captionlabeldelim{}
\newcommand*{\hm}[1]{#1\nobreak\discretionary{}

\DeclareSymbolFont{T2Aletters}{T2A}{cmr}{m}{it}
{\hbox{$\mathsurround=0pt #1$}}{}}
% Цвета для гиперссылок
\definecolor{linkcolor}{HTML}{000000} % цвет ссылок
\definecolor{urlcolor}{HTML}{799B03} % цвет гиперссылок
 
\hypersetup{pdfstartview=FitH,  linkcolor=linkcolor,urlcolor=urlcolor, colorlinks=true}


%\setcounter{secnum[utf8x]depth}{0}

\begin{document}

% НАЧАЛО ТИТУЛЬНОГО ЛИСТА
\begin{center}
  {\small ФЕДЕРАЛЬНОЕ ГОСУДАРСТВЕННОЕ АВТОНОМНОЕ ОБРАЗОВАТЕЛЬНОЕ\\ УЧРЕЖДЕНИЕ ВЫСШЕГО ОБРАЗОВАНИЯ\\ МОСКОВСКИЙ ФИЗИКО-ТЕХНИЧЕСКИЙ ИНСТИТУТ\\ (НАЦИОНАЛЬНЫЙ ИССЛЕДОВАТЕЛЬСКИЙ УНИВЕРСИТЕТ)\\ ФИЗТЕХ-ШКОЛА РАДИОТЕХНИКИ И КИБЕРНЕТИКИ}\\
  \hfill \break
  \hfill \break
  \hfill \break
  \Huge{Усилитель на биполярных транзисторах}\\
\end{center}

\hfill \break
\hfill \break
\hfill \break
\hfill \break
\hfill \break
\hfill \break

\begin{flushright}
  \normalsize{Работу выполнили:}\\
  \normalsize{\textbf{Державин Андрей \\Хайдари Фарид \\ Шурыгин Антон \\группа Б01-909}}\\
\end{flushright}

\begin{center}
  \normalsize{\textbf{Долгопрудный, 2021}}
\end{center}


\thispagestyle{empty} % выключаем отображение номера для этой страницы

% КОНЕЦ ТИТУЛЬНОГО ЛИСТА

\newpage
\thispagestyle{plain}
\tableofcontents
\thispagestyle{plain}
\newpage

\section{Основные формулы}

\[ K_u = \frac{U_{вых}}{U_{вх}} \text{\:\:\:\:} K_e = \frac{U_{вых}}{\varepsilon_{ген}} \]

\[ R_{вх} = \frac{U_{вх}}{I_{вх}} = \frac{U_{вх} R_И}{ \varepsilon_{ген} - U_{вх}} \]

Для нижней граничной частоты в случае нестабилизированного усилителя:

\[ \omega_н = \frac{1}{C_Б(R_{И} + R_{вх})} \]

В случае стабилизированного усилителя:

\[  \omega_н \approx \frac{1}{\left(\frac{R_И^{*} + h_{11}}{h_{21} + 1} || R_Э\right) \cdot C_Э}   \]

Для верхней граничной частоты в случае нестабилизированного усилителя:

\[ \omega_в = \frac{1}{ \left( \left(  C_{б1э} + C  \right) \cdot (R_{И}^{*} + r_{б1б}) || r_{б1э}     \right)  } \]

\section{Нестабилизированный усилитель}

\subsection{}

Берём радиотехнические элементы:

\begin{itemize}
  \item $R_k = 2,4 \text{ кОм}$
  \item $R_b = 540 \text{ кОм}$
  \item $R_{вх} = R_k$
\end{itemize}

Измеряем, получаем: 

\[ U_{кэ} \approx 5 \text{ B, } U_{бэ} \approx 0,64 \text{ B} \Rightarrow  \]

\[ I_{к} = \frac{U_{кэ}}{R_k} \approx 2 \text{ мА, } I_{б} = \frac{U_{вх} - U_{бэ}}{R_b} \approx  \text{17,3 мкА} \Rightarrow  \]

\[ h_{21e} \approx 115 \]

\subsection{}

Добавлем к уже имеющимся элементам:

\begin{itemize}

  \item $C = 0,47 \text{ мкФ}$
  \item $R_{и} = R_k$
\end{itemize}

Для определения $f_н$ фиксируем уменьшение $U_{вых}$ в $\sqrt{2}$ раз при переходе из области средних частот ($\approx$ 1 кГЦ) в область низких частот.
По аналогии измеряем $f_в$ при переходе из средних в высокие.
 
Результаты всех расчетов и измерений вносим в таблицу \ref{tb2}.

\section{Стабилизированный усилитель}

\subsection{}

В данном пункте считаем $h_{21э} \approx 100$. Берём радиотехнические элементы:

\begin{itemize}
  \item $R_k = 2,4 \text{ кОм}$
  \item $R_1 = 39 \text{ кОм}$
  \item $R_2 = 8,2 \text{ кОм}$
  \item $R_{и} = R_k$
  \item $R_Э = 540 \text{ Ом}$
\end{itemize}

Измеряем относительно земли напряжения, получаем:

\[ U_Б \approx 0,65 \text{ B \:\:} U_Э \approx 1,15 \text{ B \:\:} U_K \approx 5,75 \text{ B}  \]

Измеряем оставшиеся величины, заносим в таблицу. 


\[ \]

\subsection{}

\[ r_э = \frac{U_T}{I_Э} \approx 12 \text{ Ом} \]

В случае $C_Э = 0$ выполняется соотношение:

\[ K_u \approx \frac{R_k}{R_Э + r_э}\]

\[  h_{11э} \approx (h_{21э} + 1)r_э \approx 1200 \]

\[ R_Б = R_1 || R_2 \approx 6,7   \text{ кОм}\]

\[  R_{вх} = R_{Б} || (h_{11э} + R_Э (h_{21э} + 1)) \Rightarrow  \]

\[R_{вх} = \frac{R_{Б} \cdot (h_{11э} + R_Э (h_{21э} + 1)}{R_{Б} + (h_{11э} + R_Э (h_{21э} + 1)} \approx  6,8 \text{ кОм} \] 

Результаты всех расчетов и измерений вносим в таблицу \ref{tb2}.

\newpage

\section{Обработка резульатов измерений}

\begin{table}[h!]
  \centering
  \begin{tabular}{| c | c | c | c | c | c | c |}
\hline
$№ \text{\:}$ & $U_{вых \: макс}, \: В$ & $K_e$ & $K_u$ & $R_{вх}, кОм$ & $f_н, Гц$ & $f_в, МГц$\\
\hline
$\textbf{1.2}$ & $7,5$ & $71,43$ & $150$ & $2,4$ & $95$ & $0,98$\\
\hline
$\textbf{2.1}$ & $5,75$ & $4,42$ & $3,23$ & $6,5$ & $38$ & $1,1$\\
\hline
$\textbf{2.2}$ & $0,04$ & $4,59$ & $4,75$ & $6,8$ & $107$ & $0,12$\\
\hline
\end{tabular}

  \caption{}
\label{tb2}
\end{table}

%Рассчитаем и сравним теоретические значения верхней, нижней граничных частот с практикой

%\textbf{1.2}



%\textbf{2.1}



%\textbf{2.2}



\end{document}